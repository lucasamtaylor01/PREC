\documentclass[aps,column,amsmath,amssymb,floatfix]{revtex4}

\usepackage{graphicx}% Include figure files
\usepackage{dcolumn}% Align table columns on decimal point
\usepackage{bm}% bold math
\usepackage{amssymb}
\usepackage{amsmath}
\usepackage{amsfonts}
\usepackage{epsf}
\usepackage{color} % allows color in fonts
\usepackage{verbatim}
\usepackage{listings}
\usepackage{xcolor}
\usepackage{titlesec}
\usepackage{float}
\usepackage{enumitem}

\usepackage[brazilian]{babel}
\usepackage[utf8]{inputenc}
\usepackage[T1]{fontenc}
\usepackage{hyperref}
\newcommand{\PAR}[1]{\left({[#1]}\right)}


\lstdefinestyle{customc}{
  belowcaptionskip=1\baselineskip,
  breaklines=true,
  frame=none,
  xleftmargin=\parindent,
  language=C,
  showstringspaces=false,
  basicstyle=\footnotesize\ttfamily,
  keywordstyle=\bfseries\color{green!40!black},
  commentstyle=\itshape\color{purple!40!black},
  identifierstyle=\color{blue},
  stringstyle=\color{orange},
}

\lstdefinestyle{customasm}{
  belowcaptionskip=1\baselineskip,
  frame=trBL,
  xleftmargin=\parindent,
  language=[x86masm]Assembler,
  basicstyle=\footnotesize\ttfamily,
  commentstyle=\itshape\color{purple!40!black},
}

\lstset{escapechar=@,style=customc}

\titlespacing\section{0pt}{12pt plus 4pt minus 2pt}{8pt plus 2pt minus 2pt}
\titlespacing\subsection{0pt}{12pt plus 4pt minus 2pt}{8pt plus 2pt minus 2pt}
\titlespacing\subsubsection{0pt}{12pt plus 4pt minus 2pt}{0pt plus 2pt minus 2pt}

\begin{document}

\title{Relatório do EPREC - de Método Numéricos em Equações Diferenciais II}

\author{Lucas Amaral Taylor, NUSP: 13865062, graduação em Bacharelado em Matemática Aplicada e Computacional,  IME-USP}

\begin{abstract}
	\baselineskip 11pt
	Neste trabalho, aplicaremos o Método de Diferenças Finitas (MDF) para resolução numérica de equações de onda unidimensional utilizando conceitos apresentados em aula, com ênfase da utilização de derivada temporal de ordem dois.
\end{abstract}

\maketitle

\section{Introdução} 
No presente relatório, vamos estudar métodos numéricos aplicados em \textit{Python} para resolução da \textit{equação da velocidade da onda} dada por:
\begin{equation}
    u_{tt} = c^2 u_{xx}, \quad c>0.
    \label{eq:equacao-da-onda}
\end{equation}
em três problemas distintos. Em cada um deles, são fornecidas condições iniciais:
\begin{equation*}
    u(x,0) = \Phi(x) \quad \text{ e } \quad u_t(x, 0) = \Psi(x)
\end{equation*}

Para a resolução, utilizaremos o esquema de diferenças finitas dado por:
\begin{align}
   U_m^{n+1} &= c^2\lambda^2(U_{m-1}^n + U_{m+1}^n) + 2(1-c^2\lambda^2)U_m^n - U_m^{n-1} \label{eq:edf-principal}\\
   U_m^0 &= \Phi(x_m) \label{eq:edf-inicial}\\
   U_m^1 &= \frac{c^2\lambda^2}{2}(\Phi_{m-1} + \Phi_{m+1}) + (1-c^2\lambda^2)\Phi_m + \tau\Psi_m \label{eq:edf-passo-inicial}
\end{align}

As condições de fronteira consideradas são:
\begin{enumerate}[label=\roman*.]
    \item Dirichlet: $u(a,t) = \alpha(t)$ ou $u(b,t) = \beta(t)$
    \item Neumann: $u_x(a,t) = \phi(t)$ ou $u_x(b,t) = \psi(t)$
    \item Combinação das anteriores
\end{enumerate}

Utilizamos as seguintes aproximações para cada condição de fronteira:
\begin{enumerate}[label=\roman*.]
    \item Para Dirichlet:
    \begin{align*}
        \text{Se } u(a,t) &= \alpha(t): U_0^{n+1} = \alpha(t_{n+1})\\
        \text{Se } u(b,t) &= \beta(t): U_M^{n+1} = \beta(t_{n+1})
    \end{align*}   

    \item Para Neumann de ordem 1:
    \begin{align*}
        \text{Se } u_x(a,t) &= \phi(t): U_0^{n+1} = U_1^{n+1} - h\phi(t_{n+1})\\
        \text{Se } u_x(b,t) &= \psi(t): U_M^{n+1} = U_{M-1}^{n+1} + h\psi(t_{n+1})
    \end{align*}

    \item Para Neumann de ordem 2:
    \begin{align*}
        \text{Se } u_x(a,t) &= \phi(t): U_0^{n+1} = \frac{4U_1^{n+1} - U_2^{n+1} - 2h\phi(t_{n+1})}{3}\\
        \text{Se } u_x(b,t) &= \psi(t): U_M^{n+1} = \frac{4U_{M-1}^{n+1} - U_{M-2}^{n+1} + 2h\psi(t_{n+1})}{3}
    \end{align*}    
\end{enumerate}

onde:
\begin{itemize}
    \item $u(x,t)$ é a solução exata no ponto $x$ e no instante $t$
    \item $U_m^n$ é a aproximação numérica de $u(x_m,t_n)$
    \item $c$ é a velocidade da onda
    \item $h$ é o espaçamento da malha espacial
    \item $\tau$ é o espaçamento da malha temporal
    \item $\lambda = \tau/h$ é a razão entre os espaçamentos
    \item $M$ é o número de pontos da malha espacial
    \item $x_m = a + mh$ são os pontos da malha espacial
    \item $t_n = n\tau$ são os pontos da malha temporal
    \item $\Phi(x)$ é a condição inicial de posição
    \item $\Psi(x)$ é a condição inicial de velocidade
\end{itemize}

\section{Descrição da parte teórica do trabalho}
\section{Implementação e explicação da resolução da tarefa}
\section{Apresentação dos resultados}

\section{Conclusão}
\section{Referências}

\end{document}